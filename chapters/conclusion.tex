\chapter{Conclusions}

% I might put this in my acknowedgements
\begin{chapquote}{Brandon Sanderson, \emph{The Way of Kings}.}
{Journey before destination.}
\end{chapquote}

\begin{chapquote}{Wayne Dyer.}
{When you dance, your purpose is not to get to a certain place on the floor. It's to enjoy each step along the way}
\end{chapquote}

\textbf{DIPS conclusion}

DIPS, a new algorithm for identifying heavy flavour jets with impact parameter information and based on the Deep Sets architecture, has been introduced and is shown to be comparable in performance and up to a factor of 3 to 5 faster to train and evaluate over the baseline recurrent neural network based algorithm RNNIP when using the same inputs. 
The large speed-up of the algorithm facilitates optimisation, and an optimised DIPS with loosened track selections and additional per-track features was shown to improve light-flavour jet rejection by up to a factor of 2.5 and $c$-jet rejection by up to a factor of 1.5 over the baseline DIPS algorithm, which already outperforms the current RNNIP algorithm by up to 15\%. 
As such, DIPS represents a promising future direction for neural network-based flavour tagging algorithms. 
Moreover, the parallelisability and increased speed of DIPS not only has the potential to reduce the computational load of the ATLAS reconstruction, but also makes DIPS an excellent candidate for trigger applications where extremely low latency is required.
