\prefacesection{Preface}

There's something special about the human race that drives us not just to exist but to understand the reason for our existence.
A part of the mechanistic answer to this question comes from a desire to understand what are the fundamental building blocks of nature. 
As we moved away from the earth, fire, air, water understanding of the fundamental elements to a more modular undersanding of matter's components, in 400 BC, Democritus thought that he "had" it with the "elements" and he coined the word ``atoms'' to describe what we now understand as the periodic table of the elements.
Atom means ``uncuttable'' a now hilarious misnomer - as the systematic structure of the periodic table hinted at some underlying components, and the experimentally verified when Thomas X saw the electrons popping off of the atom and we came to understand the 
The evolution / progression of the scientific revolution has involved us studying ever smaller and smaller distance scales... and discovering a microcosm ever more and more intricate and facinating.
In my PhD, I've been working at SLAC the momentous institute where the substructure of the proton was discovered, a discovery which cemented our current understanding of the parton model and the Standard Model of particle phyiscs.
We keep reusing the word "elements" "elementary" because we are striving to get to these fundamental distance scales... 

Something absolutely facinating to me is the range of masses that we have in the SM that we understand to be entirely pointlike.
We currently believe that both the top quark 175 GeV  down to the < meV neutrino are entirely point-like objects, and it seems absolutely fascinating to me this 9 decade range of energies can similarly be packed into an infinitesimally small distance.
The Higgs is the answer to this mass generation - and therefore studying it and it's properties gives us a clue to this mystery.

This thesis submission marks a decade since the Higgs boson discovery, and we expect to observe the HH process in another 10 years of LHC data taking.
And although the Higgs mechaanism is now an old theory, with the massively large datasets we're collecting at the LHC, we can unlock new ways of answering these questions with the big data developments in the burgeoning field of deep learning.
This thesis explores the properties of the Higgs through the lens of the Run 2 dataset in this journey of understanding the microscopic realm -- a quest that continues to remain interesting, for a whole new generation of scientists to uncover and as we continue to ask the question... what will be next.